% Options for packages loaded elsewhere
\PassOptionsToPackage{unicode}{hyperref}
\PassOptionsToPackage{hyphens}{url}
%
\documentclass[
]{article}
\author{}
\date{\vspace{-2.5em}}

\usepackage{amsmath,amssymb}
\usepackage{lmodern}
\usepackage{iftex}
\ifPDFTeX
  \usepackage[T1]{fontenc}
  \usepackage[utf8]{inputenc}
  \usepackage{textcomp} % provide euro and other symbols
\else % if luatex or xetex
  \usepackage{unicode-math}
  \defaultfontfeatures{Scale=MatchLowercase}
  \defaultfontfeatures[\rmfamily]{Ligatures=TeX,Scale=1}
\fi
% Use upquote if available, for straight quotes in verbatim environments
\IfFileExists{upquote.sty}{\usepackage{upquote}}{}
\IfFileExists{microtype.sty}{% use microtype if available
  \usepackage[]{microtype}
  \UseMicrotypeSet[protrusion]{basicmath} % disable protrusion for tt fonts
}{}
\makeatletter
\@ifundefined{KOMAClassName}{% if non-KOMA class
  \IfFileExists{parskip.sty}{%
    \usepackage{parskip}
  }{% else
    \setlength{\parindent}{0pt}
    \setlength{\parskip}{6pt plus 2pt minus 1pt}}
}{% if KOMA class
  \KOMAoptions{parskip=half}}
\makeatother
\usepackage{xcolor}
\IfFileExists{xurl.sty}{\usepackage{xurl}}{} % add URL line breaks if available
\IfFileExists{bookmark.sty}{\usepackage{bookmark}}{\usepackage{hyperref}}
\hypersetup{
  hidelinks,
  pdfcreator={LaTeX via pandoc}}
\urlstyle{same} % disable monospaced font for URLs
\usepackage[margin=1in]{geometry}
\usepackage{graphicx}
\makeatletter
\def\maxwidth{\ifdim\Gin@nat@width>\linewidth\linewidth\else\Gin@nat@width\fi}
\def\maxheight{\ifdim\Gin@nat@height>\textheight\textheight\else\Gin@nat@height\fi}
\makeatother
% Scale images if necessary, so that they will not overflow the page
% margins by default, and it is still possible to overwrite the defaults
% using explicit options in \includegraphics[width, height, ...]{}
\setkeys{Gin}{width=\maxwidth,height=\maxheight,keepaspectratio}
% Set default figure placement to htbp
\makeatletter
\def\fps@figure{htbp}
\makeatother
\setlength{\emergencystretch}{3em} % prevent overfull lines
\providecommand{\tightlist}{%
  \setlength{\itemsep}{0pt}\setlength{\parskip}{0pt}}
\setcounter{secnumdepth}{-\maxdimen} % remove section numbering
\ifLuaTeX
  \usepackage{selnolig}  % disable illegal ligatures
\fi

\begin{document}

\hypertarget{statistical-methods-cosc-6323---home-work-9}{%
\section{Statistical Methods -- COSC 6323 -
Home-Work-9}\label{statistical-methods-cosc-6323---home-work-9}}

\hypertarget{by-dinesh-narlakanti-2083649}{%
\subsubsection{By Dinesh Narlakanti
(2083649)}\label{by-dinesh-narlakanti-2083649}}

\begin{center}\rule{0.5\linewidth}{0.5pt}\end{center}

\hypertarget{introduction}{%
\subsubsection{INTRODUCTION}\label{introduction}}

Students are divided into two groups and they are given an online talk
about an essay they wrote to a panel of three examiners. The Not
Informed (NI) group of students were not informed about the possibility
of giving a talk, while the Informed (I) group received notice already.

This document majorly focuses on:\\
i) Whether judges emotion influence the emotion of the participants.\\
ii) Is there any difference of showcasing emotions between informed
people and non informed people.\\
iii) While the stress increasing, whether the emotion is increasing or
decreasing.\\
iv) While the agreeableness of the judges increases, whether emotion of
the participants increasing or decreasing.

\hypertarget{graphs}{%
\subsubsection{GRAPHS}\label{graphs}}

\begin{verbatim}
## Data were 'prettified'. Consider using `terms="delta_pp_mean [all]"` to get smooth plots.
\end{verbatim}

\includegraphics{Report_files/figure-latex/unnamed-chunk-2-1.pdf}

\hypertarget{summary}{%
\subsubsection{SUMMARY}\label{summary}}

\begin{verbatim}
## Generalized linear mixed model fit by maximum likelihood (Laplace
##   Approximation) [glmerMod]
##  Family: binomial  ( logit )
## Formula: F_SumEmoBinary ~ (J_SumEmoBinary * Group) + (J_SumEmoBinary *  
##     delta_pp_mean) + (J_SumEmoBinary * BFI_Agreeableness) + (1 |  
##     (Participant_ID))
##    Data: mimicry
## 
##      AIC      BIC   logLik deviance df.resid 
##   4441.7   4501.0  -2211.9   4423.7     5310 
## 
## Scaled residuals: 
##      Min       1Q   Median       3Q      Max 
## -12.1280  -0.3830   0.1998   0.4976   4.5653 
## 
## Random effects:
##  Groups           Name        Variance Std.Dev.
##  (Participant_ID) (Intercept) 3.708    1.926   
## Number of obs: 5319, groups:  (Participant_ID), 38
## 
## Fixed effects:
##                                    Estimate Std. Error z value Pr(>|z|)    
## (Intercept)                       -9.421428   2.208761  -4.265 1.99e-05 ***
## J_SumEmoBinary1                   -0.713588   0.551538  -1.294 0.195729    
## GroupI                            -1.412849   0.488317  -2.893 0.003812 ** 
## delta_pp_mean                      6.028473   1.662212   3.627 0.000287 ***
## BFI_Agreeableness                  0.268834   0.054639   4.920 8.65e-07 ***
## J_SumEmoBinary1:GroupI             0.394072   0.163330   2.413 0.015833 *  
## J_SumEmoBinary1:delta_pp_mean      0.510327   0.503172   1.014 0.310478    
## J_SumEmoBinary1:BFI_Agreeableness  0.009691   0.014219   0.682 0.495525    
## ---
## Signif. codes:  0 '***' 0.001 '**' 0.01 '*' 0.05 '.' 0.1 ' ' 1
## 
## Correlation of Fixed Effects:
##             (Intr) J_SmEB1 GroupI dlt_p_ BFI_Ag J_SEB1:G J_SEB1:_
## J_SmEmBnry1 -0.146                                               
## GroupI      -0.543 -0.032                                        
## delta_pp_mn -0.381  0.106   0.250                                
## BFI_Agrblns -0.942  0.144   0.404  0.112                         
## J_SmEmB1:GI -0.016  0.086  -0.202  0.019  0.045                  
## J_SmEmB1:__  0.088 -0.614   0.027 -0.173 -0.055 -0.100           
## J_SEB1:BFI_  0.138 -0.932   0.069 -0.064 -0.157 -0.261    0.368  
## optimizer (Nelder_Mead) convergence code: 0 (OK)
## Model failed to converge with max|grad| = 0.0258967 (tol = 0.002, component 1)
## Model is nearly unidentifiable: large eigenvalue ratio
##  - Rescale variables?
\end{verbatim}

\hypertarget{observation-and-inferences}{%
\subsubsection{OBSERVATION AND
INFERENCES}\label{observation-and-inferences}}

\textbf{-\textgreater{}} From (b), Probability of showing emotion falls
from around 91\% to around 72\% for informed people. So, we can conclude
that the people who know about the presentation shows less emotion
during the presentation.

\textbf{-\textgreater{}} From (a), We can deduce that, when judges move
from neutral state to emotional state, the emotionality probability of
the participants decreased from around 91\% to around 90\%. As the
values are in opposite direction, we can say that participants show less
mimicry with the judges.

\textbf{-\textgreater{}} From (c), When judges are moving from neutral
to emotional i.e., when judges show emotion, the informed people's
probability of becoming emotional increased from around 72\% to around
76\% while probability of emotionality for non informed people decreases
from around 92\% to around 90\%. By this we can deduce that,
participants in informed group respond judges' emotions. For non
informed people, they dont mimic the judges' emotions.

\textbf{-\textgreater{}} From (d), we can clearly see that as the pp
value (stress) increasing from -0.05 to 1, the emotionality probability
of the participants also increases. So, higher the strees, higher is the
chance of showing emotion duting the presentation.

\textbf{-\textgreater{}} From (e), it is obvious that, the higher the
agreeableness, the higher is the probability of showing emotion of the
participants during the presentation. As the agreeableness score is
increasing from 20 to 45, the chance of showing emotion is also
increasing.

\textbf{-\textgreater{}} Stress of the participants influences their
expressions. We concluded this by their p-value (0.000287 ) which is
less than 0.05 and we can reject the null.

\textbf{-\textgreater{}} As p-value of agreeableness is (8.65e-07) which
is less than 0.05, we can reject null and conclude that agreeableness
score plays significant role in participants' expression.

\end{document}
